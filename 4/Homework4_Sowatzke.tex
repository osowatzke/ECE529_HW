\documentclass[fleqn]{article}
\usepackage[nodisplayskipstretch]{setspace}
\usepackage{amsmath, nccmath}
\usepackage{amssymb}
\usepackage{enumitem}
\usepackage{etoolbox}

\newcommand{\zerodisplayskip}{
	\setlength{\abovedisplayskip}{0pt}%
	\setlength{\belowdisplayskip}{0pt}%
	\setlength{\abovedisplayshortskip}{0pt}%
	\setlength{\belowdisplayshortskip}{0pt}%
	\setlength{\mathindent}{0pt}}

\makeatletter
	\newenvironment{equationCenter}{\@fleqnfalse\begin{equation*}}{\end{equation*}}
\makeatother

\title{Homework 4}
\author{Owen Sowatzke}
\date{October 24, 2023}
\setlength{\jot}{0pt}

\begin{document}
	\offinterlineskip
	\setlength{\lineskip}{12pt}
	\zerodisplayskip
	\maketitle
	\begin{enumerate}[nolistsep]
		\item[3.8] The system function of a causal LTI system is
	
		\begin{equationCenter}
			H(z) = \frac{1 - z^{-1}}{1 + \frac{3}{4}z^{-1}}
		\end{equationCenter}
		
		The input to this system is
		
		\begin{equationCenter}
			x[n] = \left(\frac{1}{3}\right)^nu[n] + u[-n-1]
		\end{equationCenter}
		
		\begin{enumerate}[nolistsep]
			\item Find the impulse response of the system, $h[n]$.
			
			We can find the impulse response, by taking the inverse z-transform of $H(z)$. However, before taking the inverse z-transform, we must first identify the ROC of $H(z)$.
			
			Because the system is casual, it must also be right-sided. Therefore, the ROC of its z-transform extends outward from
the outermost (largest magnitude) finite pole (i.e. ROC: $|z| > \frac{3}{4}$).

			If we expand the numerator of $H(z)$, we can see how it maps to the z-transform pairs in the table.
			
			\begin{equation*}
				H(z) = \frac{1}{1 + \frac{3}{4}z^{-1}} - \frac{z^{-1}}{1 + \frac{3}{4}z^{-1}},\ \text{ROC:}\ |z| > \frac{3}{4}
			\end{equation*}
			
			\begin{equation*}
				a^nu[n] \leftrightarrow \frac{1}{1 - az^{-1}},\ \text{ROC:}\ |z| > |a| 
			\end{equation*}
			
			\begin{equation*}
				a^{(n-1)}u[n-1] \leftrightarrow \frac{z^{-1}}{1 - az^{-1}},\ \text{ROC:}\ |z| > |a| 
			\end{equation*}
			
			\begin{equation*}
				\mathbf{\therefore h[n] = \left(-\frac{3}{4}\right)^nu[n] - \left(-\frac{3}{4}\right)^{(n-1)}u[n-1]}
			\end{equation*}
			
			\item Find the output $y[n]$.
				
				The z-transform of $y[n]$ can be written as follows:
				
				$Y(z) = H(z)X(z)$.
				
				We can find $X(z)$ using the table of z-transform pairs.
				
				\begin{equation*}
					a^nu[n] \leftrightarrow \frac{1}{1 - az^{-1}},\ \text{ROC:}\ |z| > |a| 
				\end{equation*}
			
				\begin{equation*}
					-u[-n-1] \leftrightarrow \frac{1}{1 - z^{-1}},\ \text{ROC:}\ |z| < 1 
				\end{equation*}
				
				\begin{equation*}
					\therefore X(z) = \frac{1}{1 - \frac{1}{3}z^{-1}} - \frac{1}{1 - z^{-1}} = \frac{1 - z^{-1} - \left(1 - \frac{1}{3}z^{-1}\right)}{\left(1 - \frac{1}{3}z^{-1}\right)\left(1 - z^{-1}\right)}
				\end{equation*}
				
				\begin{equation*}
					= \frac{-\frac{2}{3}z^{-1}}{\left(1 - \frac{1}{3}z^{-1}\right)\left(1 - z^{-1}\right)},\ \text{ROC:}\ \frac{1}{3} < |z| < 1 
				\end{equation*}
				
				\begin{equation*}
					Y(z) = \frac{-\frac{2}{3}z^{-1}(1 - z^{-1})}{\left(1 - \frac{1}{3}z^{-1}\right)\left(1 - z^{-1}\right)\left(1 + \frac{3}{4}z^{-1}\right)}
				\end{equation*}
				
				\begin{equation*}
					= \frac{-\frac{2}{3}z^{-1}}{\left(1 - \frac{1}{3}z^{-1}\right)\left(1 + \frac{3}{4}z^{-1}\right)},\ \text{ROC:}\ |z| > \frac{3}{4} 
				\end{equation*}
				
				Use partial fraction expansion to rewrite $Y(z)$ in a form that we can apply z-transform pairs to.
				
				\begin{equation*}
					\frac{-\frac{2}{3}z^{-1}}{\left(1 - \frac{1}{3}z^{-1}\right)\left(1 + \frac{3}{4}z^{-1}\right)} = \frac{A}{1 - \frac{1}{3}z^{-1}} + \frac{B}{1 + \frac{3}{4}z^{-1}}
				\end{equation*}
				
				\begin{equation*}
					A = \left.\frac{-\frac{2}{3}z^{-1}}{1 + \frac{3}{4}z^{-1}}\right\vert_{z = \frac{1}{3}} = \frac{-\frac{2}{3}(3)}{1 + \frac{3}{4}(3)} = -\frac{2}{1 + \frac{9}{4}} = -\frac{8}{4 + 9} = -\frac{8}{13}
				\end{equation*}
					
				\begin{equation*}
					B = \left.\frac{-\frac{2}{3}z^{-1}}{1 - \frac{1}{3}z^{-1}}\right\vert_{z = -\frac{3}{4}} = \frac{-\frac{2}{3}\left(-\frac{4}{3}\right)}{1 - \frac{1}{3}\left(-\frac{4}{3}\right)} = \frac{\frac{8}{9}}{1 + \frac{4}{9}} = \frac{8}{4 + 9} = \frac{8}{13}
				\end{equation*}
				
				\pagebreak
				\begin{equation*}
					Y(z) = \frac{-\frac{8}{13}}{1 - \frac{1}{3}z^{-1}} + \frac{\frac{8}{13}}{1 + \frac{3}{4}z^{-1}},\ \text{ROC:}\ |z| > \frac{3}{4}
				\end{equation*}
				
				We can now use the following z-transform pair to derive $y[n]$:
				
				\begin{equation*}
					a^nu[n] \leftrightarrow \frac{1}{1 - az^{-1}},\ \text{ROC:}\ |z| > |a| 
				\end{equation*}
				
				\begin{equation*}
					\mathbf{\therefore y[n] = -\frac{8}{13}\left(\frac{1}{3}\right)^nu[n] + \frac{8}{13}\left(-\frac{3}{4}\right)^nu[n]}
				\end{equation*}
				
				
			\item Is the system stable? That is, is $h[n]$ absolutely summable?
			
				The ROC of $H(z)$ includes the unit circle.
				
				\textbf{$\mathbf{\therefore}$ the system is stable.}
		\end{enumerate}
		
				
		\item [3.10] Without explicitly solving for $X(z)$ find the ROC of the z-transform of each of the following sequences, and determine whether the Fourier transform converges:
		
			\begin{enumerate}[nolistsep]
				\item [(b)]

					\begin{equation*}				
						x[n] = \left\{\begin{aligned} 1,& \ -10 \leq n \leq 10, \\ 
						0,& \ \text{otherwise}, \end{aligned}\right.
					\end{equation*}
					
					$x[n]$ is a finite duration signal, so its ROC will be all of $z$ except possibly $z = 0$ or $z = \infty$.
					
					The non-zero $x[n]$ samples for $n > 0$ will create poles at $0$, and the non-zero $x[n]$ samples for $n < 0$ will create poles at $\infty$.
					
					\textbf{$\mathbf{\therefore}$ the ROC of the z-transform will be all of $\mathbf{z}$ except for $\mathbf{z=0}$ and $\mathbf{z=\infty}$. Because the unit circle is contained within the ROC of $\mathbf{X(z)}$, the Fourier Transform converges.}
					
				\item[(f)] $x[n] = \left(\frac{1}{2}\right)^{n-1}u[n] + (2 + 3j)^{n-2}u[-n-1]$
				
				Let $X_1(z) = \left(\frac{1}{2}\right)^{n-1}u[n]$
				
				$X_1(z) \propto \left(\frac{1}{2}\right)^{n}u[n]$.
				
				$\therefore X_1(z)$ has a ROC of $|z| > \frac{1}{2}$
				
				Let $X_2(z) = (2 + 3j)^{n-2}u[-n-1]$
				
				$X_2(z) \propto (2 + 3j)^{n}u[-n-1]$.
				
				$\therefore X_2(z)$ has a ROC of $|z| < |2 + 3j| = \sqrt{13}$
				
				If we denote $R_{X_1}$ as the ROC of $X_1(z)$ and $R_{X_2}$ as the ROC of $X_2(z)$, the ROC of $X_1(z) + X_2(z)$ contains $R_{X_1} \cap R_{X_2}$. If there is no pole-zero cancellation, the ROC of $X_1(z) + X_2(z)$ is exactly $R_{X_1} \cap R_{X_2}$.
				
				If we were to simplify the expressions for $X_1(z) + X_2(z)$, we would find the zeros of the expression and conclude that there is no pole-zero cancellation.
				
				\textbf{$\mathbf{\therefore}$ the ROC of the z-transform will be $\mathbf{\frac{1}{2} < |z| < \sqrt{13}}$. Because the unit circle is contained within the ROC of $\mathbf{X(z)}$, the Fourier Transform converges.}
			\end{enumerate}
			
		\item[3.21] A causal LTI system has the following system function:
		
			\begin{equationCenter}
				H(z) = \frac{4 + 0.25z^{-1}-0.5z^{-2}}{(1-0.25z^{-1})(1 + 0.5z^{-1})}
			\end{equationCenter}
		
			\begin{enumerate}[nolistsep]
				\item[(a)] What is the ROC for $H(z)$?
				
					Because the system is casual, it must also be right-sided. Therefore, the ROC of its z-transform extends outward from
the outermost (largest magnitude) finite pole.

					\textbf{$\mathbf{\therefore}$ ROC is $\mathbf{|z| > 0.5}$}
					
				\item[(b)] Determine if the system is stable or not.
					The ROC of $H(z)$ includes the unit circle.
				
					\textbf{$\mathbf{\therefore}$ the system is stable.}
				 
				 \item[(c)] Determine the difference equation that is satisfied by the input $x[n]$ and the output $y[n]$.
				 
				 	\begin{equation*}
				 		H(z) = \frac{Y(z)}{X(z)} = \frac{4 + 0.25z^{-1}-0.5z^{-2}}{(1-0.25z^{-1})(1 + 0.5z^{-1})}
				 	\end{equation*}
				 	
				 	$Y(z)(1-0.25z^{-1})(1 + 0.5z^{-1}) = X(z)(4 + 0.25z^{-1}-0.5z^{-2})$
				 	
				 	$Y(z)(1 + 0.25z^{-1} - 0.125z^{-2}) = X(z)(4 + 0.25z^{-1}-0.5z^{-2})$
				 	
				 	$Y(z) + 0.25Y(z)z^{-1} - 0.125Y(z)z^{-2}$
				 	
				 	$= 4X(z) + 0.25X(z)z^{-1}-0.5X(z)z^{-2}$
				 	
				 	Take the inverse z-transform of both sides of the equation.
				 	
				 	$y[n] + 0.25y[n-1] - 0.125y[n-2] = 4x[n] + 0.25x[n-1] - 0.5x[n-2]$
				 	
				 	Rearrange the terms to obtain a difference equation for $y[n]$
				 	
				 	$\mathbf{y[n] = -0.25y[n-1] + 0.125y[n-2]}$
				 	
				 	$\mathbf{ + 4x[n] + 0.25x[n-1] - 0.5x[n-2]}$
			\end{enumerate}
	\end{enumerate}
	
\end{document}