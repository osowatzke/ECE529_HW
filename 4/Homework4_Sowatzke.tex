\documentclass[fleqn]{article}
\usepackage[nodisplayskipstretch]{setspace}
\usepackage{amsmath, nccmath}
\usepackage{amssymb}
\usepackage{enumitem}
\usepackage{etoolbox}

\newcommand{\zerodisplayskip}{
	\setlength{\abovedisplayskip}{0pt}%
	\setlength{\belowdisplayskip}{0pt}%
	\setlength{\abovedisplayshortskip}{0pt}%
	\setlength{\belowdisplayshortskip}{0pt}%
	\setlength{\mathindent}{0pt}}

\makeatletter
	\newenvironment{equationCenter}{\@fleqnfalse\begin{equation*}}{\end{equation*}}
\makeatother

\title{Homework 4}
\author{Owen Sowatzke}
\date{October 24, 2023}
\setlength{\jot}{0pt}

\begin{document}
	\offinterlineskip
	\setlength{\lineskip}{12pt}
	\zerodisplayskip
	\maketitle
	\begin{enumerate}[nolistsep]
		\item[3.8] The system function of a causal LTI system is
	
		\begin{equationCenter}
			H(z) = \frac{1 - z^{-1}}{1 + \frac{3}{4}z^{-1}}
		\end{equationCenter}
		
		The input to this system is
		
		\begin{equationCenter}
			x[n] = \left(\frac{1}{3}\right)^nu[n] + u[-n-1]
		\end{equationCenter}
		
		\begin{enumerate}[nolistsep]
			\item Find the impulse response of the system, $h[n]$.
			
			We can find the impulse response, by taking the inverse z-transform of $H(z)$. However, before taking the inverse z-transform, we must first identify the ROC of $H(z)$.
			
			Because the system is casual, it must also be right-sided. Therefore, the ROC of its z-transform extends outward from
the outermost (largest magnitude) finite pole (i.e. ROC: $|z| > \frac{3}{4}$).

			If we expand the numerator of $H(z)$, we can see how it maps to the z-transform pairs in the table.
			
			\begin{equation*}
				H(z) = \frac{1}{1 + \frac{3}{4}z^{-1}} - \frac{z^{-1}}{1 + \frac{3}{4}z^{-1}},\ \text{ROC:}\ |z| > \frac{3}{4}
			\end{equation*}
			
			\begin{equation*}
				a^nu[n] \leftrightarrow \frac{1}{1 - az^{-1}},\ \text{ROC:}\ |z| > |a| 
			\end{equation*}
			
			\begin{equation*}
				a^{(n-1)}u[n-1] \leftrightarrow \frac{z^{-1}}{1 - az^{-1}},\ \text{ROC:}\ |z| > |a| 
			\end{equation*}
			
			\begin{equation*}
				\mathbf{\therefore h[n] = \left(-\frac{3}{4}\right)^nu[n] - \left(-\frac{3}{4}\right)^{(n-1)}u[n-1]}
			\end{equation*}
			
			\item Find the output $y[n]$.
				
				The z-transform of $y[n]$ can be written as follows:
				
				$Y(z) = H(z)X(z)$.
				
				We can find $X(z)$ using the table of z-transform pairs.
				
				\begin{equation*}
					a^nu[n] \leftrightarrow \frac{1}{1 - az^{-1}},\ \text{ROC:}\ |z| > |a| 
				\end{equation*}
			
				\begin{equation*}
					-u[-n-1] \leftrightarrow \frac{1}{1 - z^{-1}},\ \text{ROC:}\ |z| < 1 
				\end{equation*}
				
				\begin{equation*}
					\therefore X(z) = \frac{1}{1 - \frac{1}{3}z^{-1}} - \frac{1}{1 - z^{-1}} = \frac{1 - z^{-1} - \left(1 - \frac{1}{3}z^{-1}\right)}{\left(1 - \frac{1}{3}z^{-1}\right)\left(1 - z^{-1}\right)}
				\end{equation*}
				
				\begin{equation*}
					= \frac{-\frac{2}{3}z^{-1}}{\left(1 - \frac{1}{3}z^{-1}\right)\left(1 - z^{-1}\right)},\ \text{ROC:}\ \frac{1}{3} < |z| < 1 
				\end{equation*}
				
				\begin{equation*}
					Y(z) = \frac{-\frac{2}{3}z^{-1}(1 - z^{-1})}{\left(1 - \frac{1}{3}z^{-1}\right)\left(1 - z^{-1}\right)\left(1 + \frac{3}{4}z^{-1}\right)}
				\end{equation*}
				
				\begin{equation*}
					= \frac{-\frac{2}{3}z^{-1}}{\left(1 - \frac{1}{3}z^{-1}\right)\left(1 + \frac{3}{4}z^{-1}\right)},\ \text{ROC:}\ |z| > \frac{3}{4} 
				\end{equation*}
				
				Use partial fraction expansion to rewrite $Y(z)$ in a form that we can apply z-transform pairs to.
				
				\begin{equation*}
					\frac{-\frac{2}{3}z^{-1}}{\left(1 - \frac{1}{3}z^{-1}\right)\left(1 + \frac{3}{4}z^{-1}\right)} = \frac{A}{1 - \frac{1}{3}z^{-1}} + \frac{B}{1 + \frac{3}{4}z^{-1}}
				\end{equation*}
				
				\begin{equation*}
					A = \left.\frac{-\frac{2}{3}z^{-1}}{1 + \frac{3}{4}z^{-1}}\right\vert_{z = \frac{1}{3}} = \frac{-\frac{2}{3}(3)}{1 + \frac{3}{4}(3)} = -\frac{2}{1 + \frac{9}{4}} = -\frac{8}{4 + 9} = -\frac{8}{13}
				\end{equation*}
					
				\begin{equation*}
					B = \left.\frac{-\frac{2}{3}z^{-1}}{1 - \frac{1}{3}z^{-1}}\right\vert_{z = -\frac{3}{4}} = \frac{-\frac{2}{3}\left(-\frac{4}{3}\right)}{1 - \frac{1}{3}\left(-\frac{4}{3}\right)} = \frac{\frac{8}{9}}{1 + \frac{4}{9}} = \frac{8}{4 + 9} = \frac{8}{13}
				\end{equation*}
				
				\pagebreak
				\begin{equation*}
					Y(z) = \frac{-\frac{8}{13}}{1 - \frac{1}{3}z^{-1}} + \frac{\frac{8}{13}}{1 + \frac{3}{4}z^{-1}},\ \text{ROC:}\ |z| > \frac{3}{4}
				\end{equation*}
				
				We can now use the following z-transform pair to derive $y[n]$:
				
				\begin{equation*}
					a^nu[n] \leftrightarrow \frac{1}{1 - az^{-1}},\ \text{ROC:}\ |z| > |a| 
				\end{equation*}
				
				\begin{equation*}
					\mathbf{\therefore y[n] = -\frac{8}{13}\left(\frac{1}{3}\right)^nu[n] + \frac{8}{13}\left(-\frac{3}{4}\right)^nu[n]}
				\end{equation*}
				
				
			\item Is the system stable? That is, is $h[n]$ absolutely summable?
		\end{enumerate}
	\end{enumerate}
	
\end{document}