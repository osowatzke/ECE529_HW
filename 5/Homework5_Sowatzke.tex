\documentclass[fleqn]{article}
\usepackage[nodisplayskipstretch]{setspace}
\usepackage{amsmath, nccmath}
\usepackage{amssymb}
\usepackage{enumitem}
\usepackage{matlab-prettifier}
\usepackage{scalerel}
\usepackage{graphicx}
\usepackage{float}
\usepackage{changepage}
\usepackage{environ,capt-of}

\newcommand{\zerodisplayskip}{
	\setlength{\abovedisplayskip}{0pt}%
	\setlength{\belowdisplayskip}{0pt}%
	\setlength{\abovedisplayshortskip}{0pt}%
	\setlength{\belowdisplayshortskip}{0pt}%
	\setlength{\mathindent}{0pt}}
	
\let\oldfigure\figure% Store original figure float environment
\let\endoldfigure\endfigure
\RenewEnviron{figure}[1][H]{% Update figure environment
  %\par\vspace{\intextsep}% Assume in-text placement, so insert appropriate vertical spacing
  \noindent
  % \patchcmd{<cmd>}{<search>}{<replace>}{<success>}{<failure>}
  \patchcmd{\BODY}{\caption}{\captionof{figure}}{}{}% Replace \caption with \captionof{figure} inside \BODY
  % Set "figure"
  \begin{minipage}{\linewidth}
    \BODY
  \end{minipage}
  %\par\vspace{\intextsep}% Assume in-text placement, so insert appropriate vertical spacing
}

\title{Homework 5}
\author{Owen Sowatzke}
\date{November 16, 2023}

\begin{document}
	\offinterlineskip
	\setlength{\lineskip}{12pt}
	\zerodisplayskip
	\maketitle
	
	\begin{enumerate}[nolistsep]
		\item Design a discrete-time second-order Butterworth lowpass filter with cutoff frequency equivalent to $f_c = 1.6 \text{kHz}$. Assume the ADC sampling frequency is $f_s = 8 \text{kHz}$. Use the impulse invariance method with $T_d = 1$.
		
			\begin{enumerate}[nolistsep]
				\item Calculate the cutoff frequency for the desired digital filter (i.e., convert the analog specs to digital specs.)
				
					\begin{equation*}
						\omega_c = \frac{2{\pi}f_c}{f_s} = \frac{2{\pi}(1.6\ \text{kHz})}{8\ \text{kHz}} = \mathbf{0.4\pi}
					\end{equation*}
					
				\item Calculate the cutoff frequency for the prototype analog filter.
				
					\begin{equation*}
						\Omega_c = \frac{\omega_c}{T_d} = \mathbf{0.4\pi} 
					\end{equation*}
					
				\item Derive the transfer function $H_a(s)$ for the prototype analog filter by selecting the left-half-plane poles of $H_a(s)H_a(-s)$. Show your work. Show that this matched the result of the following MATLAB command:
				
				\texttt{[bs, as] = butter(N,$\Omega_c$,'low','s')}
				
				\begin{equation*}
					H_a(s)H_a(-s) = \frac{(-1)^N{\Omega_c}^{2N}}{s^{2N} + (-1)^N{\Omega_c}^{2N}}
				\end{equation*}
				
				\begin{equation*}
					= \frac{(-1)^2(0.4\pi)^4}{s^4 + (-1)^4(0.4\pi)^4} = \frac{(0.4\pi)^4}{s^4 + (0.4\pi)^4}
				\end{equation*}
				
				Solve for the poles of the above expression:
				
				$s^4 + (0.4\pi)^4 = 0$
				
				$\Rightarrow s^4 = -(0.4\pi)^4$
				
				$\Rightarrow s^4 = (0.4\pi)^4e^{j\pi}$
				
				$\Rightarrow s^4 = (0.4\pi)^4e^{j(\pi + 2{\pi}k)}\ \forall\ k \in \mathbb{Z}$
				
				$\Rightarrow s = 0.4{\pi}e^{j({\pi/4} + {\pi}k/2)}\ \forall\ k \in \mathbb{Z}$
				
				$\therefore$ $H_a(s)H_a(-s)$ has 4 unique poles:
				
				$s_1 = 0.4{\pi}e^{j{\pi}/4}$
				
				$s_2 = 0.4{\pi}e^{j3{\pi}/4}$
				
				$s_3 = 0.4{\pi}e^{-j3{\pi}/4}$
				
				$s_4 = 0.4{\pi}e^{-j{\pi}/4}$
				
				Because the filter is stable, the poles of $H_a(s)$ must be in the LHP.
				
				$\therefore$ $H_a(s)$ has two poles:
				
				$s_1 = 0.4{\pi}e^{j3{\pi}/4}$
				
				$s_2 = 0.4{\pi}e^{-j3{\pi}/4}$
				
				Now, we can rewrite $H_a(s)$ in the following form:
				
				\begin{equation*}
					H_a(s) = \frac{(0.4\pi)^2}{(s - 0.4{\pi}e^{j3{\pi}/4})(s - 0.4{\pi}e^{-j3{\pi}/4})}
				\end{equation*}
				
				\begin{equation*}
					= \frac{(0.4\pi)^2}{s^2 - (0.4{\pi}e^{j3{\pi}/4} + 0.4{\pi}e^{-j3{\pi}/4})s + (0.4\pi)^2}
				\end{equation*}
				
				\begin{equation*}
					= \frac{(0.4\pi)^2}{s^2 - 0.8\pi\cos(3\pi/4)s + (0.4\pi)^2}
				\end{equation*}
				
				\begin{equation*}
					= \frac{1.5791}{s^2 + 1.7772s + 1.5791}
				\end{equation*}
				
				\pagebreak
				We can compare this transfer function to the output of the following MATLAB command:
				
				\begin{figure}[H]
					\centerline{\fbox{\includegraphics[width=0.5\textwidth]{prob_1c.png}}}
					\caption{Using MATLAB to Solve for $H_a(s)$}
				\end{figure}
				
				The transfer function $H_a(s)$ derived using MATLAB has numerator polynomial of $0s^2 + 0s + 1.5791 = 1.5971$ and a denominator polynomial of $s^2 + 1.7772s + 1.5791$. Note that this is equivalent to the transfer function that was analytically derived.
				
				%Running:
				
				%\texttt{[bs, as] = butter(N,0.4*pi,'low','s')}
				
				%we get:
				
				%\texttt{bs = [0}
				%After
				
				\item Derive the transfer function $H(z)$ for the desired digital filter. Also show the result of the following MATLAB command, which gives the coefficients for the final, simplified $H(z)$:
				
				\texttt{[bz, az] = impinvar(bs,as,$1/T_d$)}
				
				We first need to find $h_a(t)$ by taking an inverse Laplace transform of $H_a(s)$. Start by using partial fraction expansion to write $H_a(s)$ in the following form:
				
				\begin{equation*}
					H_a(s) = \frac{A_1}{s - 0.4{\pi}e^{j3{\pi}/4}} +  \frac{A_2}{s - 0.4{\pi}e^{-j3{\pi}/4}}
				\end{equation*}
				
				\begin{equation*}
					A_1 = H_a(s)(s - 0.4{\pi}e^{j3{\pi}/4}) = \left.\frac{(0.4\pi)^2}{s - 0.4{\pi}e^{-j3{\pi}/4}}\right\vert_{s = 0.4{\pi}e^{j3{\pi}/4}}
				\end{equation*}
				
				\begin{equation*}
					= \frac{(0.4\pi)^2}{0.4{\pi}e^{j3{\pi}/4} - 0.4{\pi}e^{-j3{\pi}/4}} = \frac{(0.4\pi)^2}{j0.8{\pi}\sin(3\pi/4)}
				\end{equation*}
				
				\begin{equation*}
					 = \frac{(0.4\pi)}{j2(1/\sqrt{2})} = -\frac{j\pi\sqrt{2}}{5}
				\end{equation*}
				
				\begin{equation*}
					A_2 = H_a(s)(s - 0.4{\pi}e^{-j3{\pi}/4}) = \left.\frac{(0.4\pi)^2}{s - 0.4{\pi}e^{j3{\pi}/4}}\right\vert_{s = 0.4{\pi}e^{-j3{\pi}/4}}
				\end{equation*}
				
				\begin{equation*}
					= \frac{(0.4\pi)^2}{0.4{\pi}e^{-j3{\pi}/4} - 0.4{\pi}e^{j3{\pi}/4}} = \frac{(0.4\pi)^2}{-j0.8{\pi}\sin(3\pi/4)}
				\end{equation*}
				
				\begin{equation*}
					 = \frac{(0.4\pi)}{-j2(1/\sqrt{2})} = \frac{j\pi\sqrt{2}}{5}
				\end{equation*}
				
				\begin{equation*}
					H_a(s) = \frac{-j\pi\sqrt{2}}{5}\left(\frac{1}{s - 0.4{\pi}e^{j3\pi/4}}\right) + \frac{j\pi\sqrt{2}}{5}\left(\frac{1}{s - 0.4{\pi}e^{-j3\pi/4}}\right)
				\end{equation*}
				
				\begin{equation*}
					= \frac{-j\pi\sqrt{2}}{5}\left(\frac{1}{s - 0.4{\pi}\left(-\frac{1}{\sqrt{2}} + \frac{j}{\sqrt{2}}\right)}\right)
				\end{equation*}
				
				\begin{equation*}
					 + \frac{j\pi\sqrt{2}}{5}\left(\frac{1}{s - 0.4{\pi}\left(-\frac{1}{\sqrt{2}} - \frac{j}{\sqrt{2}}\right)}\right)
				\end{equation*}
				
				Now take the inverse Laplace transform to get $h_a(t)$.
				
				\begin{equation*}
					h_a(t) = \frac{-j\pi\sqrt{2}}{5}e^{0.4\pi\left(-\frac{1}{\sqrt{2}} + \frac{j}{\sqrt{2}}\right)t} + \frac{j\pi\sqrt{2}}{5}e^{0.4\pi\left(-\frac{1}{\sqrt{2}} - \frac{j}{\sqrt{2}}\right)t}
				\end{equation*}
				
				\begin{equation*}
					= \frac{\pi\sqrt{2}}{5}e^{\frac{j0.4{\pi}t}{\sqrt{2}}}\left(-je^{\frac{j0.4{pi}t}{\sqrt{2}}} + je^{-j\frac{0.4{\pi}t}{\sqrt{2}}}\right)
				\end{equation*}
				
				\begin{equation*}
					= \frac{2\pi\sqrt{2}}{5}e^{\frac{j0.4{\pi}t}{\sqrt{2}}}\left(\frac{e^{\frac{j0.4{\pi}t}{\sqrt{2}}} - e^{-\frac{j0.4{\pi}t}{\sqrt{2}}}}{j2}\right)
				\end{equation*}
				
				\begin{equation*}
					 = \frac{2\pi\sqrt{2}}{5}e^{\frac{j0.4{\pi}t}{\sqrt{2}}}\sin{\left(\frac{0.4{\pi}t}{\sqrt{2}}\right)}
				\end{equation*}
			\end{enumerate}
	\end{enumerate}
\end{document}
