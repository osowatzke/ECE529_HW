\documentclass[fleqn]{article}

\usepackage{geometry}
\usepackage{amsmath, nccmath}
\usepackage{amssymb}
\usepackage{graphicx}
\usepackage{enumitem}
\usepackage[nodisplayskipstretch]{setspace}
\usepackage{float}

\title{Homework 1}
\author{Owen Sowatzke}
\date{September 12, 2023}

\begin{document}

	\setlength{\abovedisplayskip}{0pt}
	\setlength{\belowdisplayskip}{0pt}
	\setlength{\abovedisplayshortskip}{0pt}
	\setlength{\belowdisplayshortskip}{0pt}
	\setlength{\mathindent}{0pt}
	\doublespacing
	\maketitle
	
	\begin{enumerate}[nolistsep]
	
		\item[2.7] Determine whether each of the following signals is periodic. If the signal is periodic, state its period.
		
		\begin{enumerate}[nolistsep]
			
			\item[(b)] $x[n] = e^{j({3\pi}n/4)}$
			
			\begin{align*}
			\frac{\omega}{2\pi} = \frac{k}{N}
			\end{align*}
			
			\begin{align*}
			\frac{3\pi/4}{2\pi} = \frac{k}{N}
			\end{align*}
			
			\begin{align*}
			\frac{3}{8} = \frac{k}{N}
			\end{align*}
			
			Because $\frac{k}{N}$ is rational, the signal is periodic, and the period of the signal is $N = 8$.
			
			\item[(d)] $x[n] = e^{j{\pi}n/\sqrt{2}}$
			
			\begin{align*}
			\frac{\omega}{2\pi} = \frac{k}{N}
			\end{align*}
			
			\begin{align*}
			\frac{\pi/\sqrt{2}}{2\pi} = \frac{k}{N}
			\end{align*}
			
			\begin{align*}
			\frac{1}{2\sqrt{2}} = \frac{k}{N}
			\end{align*}
			
			Because $\frac{k}{N}$ is irrational, the signal is not periodic.
			
		\end{enumerate}
		\item[2.21] A discrete-time signal $x[n]$ is shown in Figure \ref{prob_statement}.
		
		%\renewcommand{\thefigure}{P2.21}	
	
		\begin{figure}[H]		\centerline{\fbox{\includegraphics[width=0.35\textwidth]{p2.21_problem_statement.png}}}
		\caption{Discrete Time Signal $x[n]$}
		\label{prob_statement}
		\end{figure}
		
		Sketch and label carefully each of the following signals:
		
		\begin{enumerate}[nolistsep]
			\item[(c)] $x[2n]$
			
			%\renewcommand{\thefigure}{\arabic{figure}}
			%\setcounter{figure}{0}
			
			\begin{figure}[H]				
			\centerline{\fbox{\includegraphics[width=0.35\textwidth]{p2.21_c.png}}}
		\caption{Discrete Time Signal $x[2n]$}
		\label{part_c}
		\end{figure}
		
			\item[(e)] $x[n - 1]\delta[n - 3]$
			
			\begin{figure}[H]
			\centerline{\fbox{\includegraphics[width=0.35\textwidth]{p2.21_e_x_nm1.png}}}
		\caption{Discrete Time Signal $x[n-1]$}
		\label{part_e_x_nm1}
		\end{figure}
			
			\begin{figure}[H]
			\centerline{\fbox{\includegraphics[width=0.35\textwidth]{p2.21_e_delta_nm3.png}}}
		\caption{Discrete Time Signal $\delta[n-3]$}
		\label{part_e_delta_nm3}
		\end{figure}
		
		\begin{figure}[H]
			\centerline{\fbox{\includegraphics[width=0.35\textwidth]{p2.21_e.png}}}
		\caption{Discrete Time Signal $x[n-1]\delta[n-3]$}
		\label{part_e}
		\end{figure}
		
		\end{enumerate}
		
		
	\end{enumerate}
\end{document}